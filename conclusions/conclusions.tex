% 	HTML5 Robot User Interface Project Report: Conclusions
% 	An ASLab Project,
% 	Developed by Daniel Peiró
% 	ETSII, UPM 2014-2015
\chapter{Conclusions} \label{conclusions}
This chapter briefly analyzes the success and failures of the project. The project objectives and requirements are clearly 
outlined in the introductory chapter \ref{introduction}, and subsequent chapters have thoroughly discussed the technologies 
used and architectures implemented to attain the final product.\\

The requirements are laid out the same way as in the introduction, and the success in complying with them is discussed.
\begin{itemize}
	\item \textbf{Universal}: \textit{The GUI must be easily adaptable for use in a wide array of robots, with different 
	features and lack thereof}.\\

	HRUI has been shown to be easily adaptable to robots with different features, through the 3 proof-of-concept 
	integrations documented (see \ref{vrepvirtualkheperaiii}, \ref{kheperaIII} and \ref{crazyflie2}), and the undocumented 
	fourth integration with the Parrot Rolling Spider.\\

	\item \textbf{Accessible}: \textit{This is a two-fold requirement}:
	\begin{itemize}
		\item \textit{The GUI must be easily usable from a wide array of devices, including but not limited to desktop and 
		laptop computers, mobile devices such as smartphones and tablets, all running different operating systems, which 
		the application should be agnostic of}.\\

		The GUI has been shown to work on Desktop and laptop computers as well as smartphones and tablets. The GUI is 
		completely agnostic of the client-host setup, given that it's run inside the browser environment.\\

		\item \textit{The application that provides the GUI must also expose an API (Application Programming Interface) 
		that enables external developers to use the inputs and outputs from the GUI, so that the first requirement can be 
		truly attained, by adapting them to whatever application necessary. This interface should be as simple as possible, 
		and comply with all other requirements, where applicable (i.e. it must be portable, distributed, real-time capable 
		and open source)}.\\

		The API exposed by HRUI is accessible from almost any modern general purpose programming language through the 
		mongoDB drivers that are available (see section \ref{mongodb}). It's as simple as a one-liner with a 3 line setup 
		for Python (see section \ref{python}), and is similarly simple for all other languages. It's portable, because 
		mongoDB is multi-platform; it's distributed, because mongoDB allows network connection to the database; it's 
		real-time, given mongoDB's blazing fast r/w speed; and is of course open source, given that HRUI can be modified 
		easily and mongoDB is open source software (see section \ref{opensourcemovement} for more).
	\end{itemize}
	\item \textbf{Customizable}: \textit{Partially as a result of the first requirement, the GUI must have a variety of 
	inputs and outputs that can be used independently from one another, so that multiple configurations can be setup and 
	used for different robots, and different applications of the same robot. If possible, the user should be able to tailor 
	custom inputs and outputs when the proposed tools are insufficient for the particular application. This process should 
	be seamless and transparent for the user, when possible}.\\

	The modular front-end architecture (see section \ref{modularfrontendarchitecture}) allows any configuration of modules 
	to be saved and retrieved for future use (see section \ref{profilemanagement} on the profile management module). The 
	Custom Input (section \ref{custominput}) and Custom Data (section \ref{customdata}) modules allow the user to create as 
	many inputs and outputs as necessary on the fly, just by declaring their names and types. This is done without even 
	having to reload the interface, and allowing the user to save the inputs and outputs for future retrieval using the 
	mentioned profile management module. The retrieval on the back-end of the inputs and the writing of the outputs is as 
	simple as using the mongoDB API.

	\item \textbf{Distributed}: \textit{Partially as a result of the accessibility requirement, the application must be 
	distributed over a network, in such a way that the GUI need not be physically tethered to a particular machine, or 
	locally executed for the application to function.}\\

	HRUI is a distributed application that can communicate the client with the server (see section 
	\ref{clientserverpattern}) over a local area network or over the Internet with minimal to no input lag. This has been 
	tested using even a cellular Internet connection. Furthermore, the mongoDB API is distributed, so the back-end is also 
	distributed regarding the robot controller.

	\item \textbf{Real-Time}: \textit{The inputs registered by the GUI must be able to control the robot in real-time, as 
	well as receive output from the robot in real-time. No hard real-time deadlines are set, given that the application is 
	distributed, these deadlines would be hard to estimate without extensive research and control over the environment of 
	the application, hence negating the accessibility requirement. However, the response time of the system should be 
	maximized to the extent of the current state of the art of the technologies used.}\\

	HRUI is a real-time application with inputs being available as fast as the events that generate them are produced, with 
	a maximum delay of a few milliseconds (the actual value has not been measured, for lack of appropriate tools, but is 
	surely well within the bounds of reasonable soft real-time constraints). Output presentation is delayed purposefully to 
	make the front-end more efficient so that low-end hardware can run it comfortably, but the same speed is available if 
	so required.

	\item \textbf{Portable}: \textit{The application that provides the GUI must be deployable on a wide array of systems 
	with regards to these variables:}\\
		\begin{itemize}
			\item Hardware Capabilities: \textit{The application must be deployable on portable, affordable, low-end 
			computers, as to be easily mountable on mobile robots if so required.}\\

			HRUI is deployable on a Raspberry Pi 2, a 35\$ micro computer with a 900MHz quad-core processor and 1GB of RAM 
			Memory. See section \ref{raspberrypi2} for more.


			\item Operating System: \textit{The application must be multi-platform, meaning it can be deployed on a variety 
			of operating systems. These OSs must include at least, but not necessarily limited to: a general purpose Linux 
			distribution, Microsoft Windows and Apple OS X. Only the latest versions of each need to be supported. Required 
			Architectures, though not necessarily limited to: 32/64-bit x86}.\\

			HRUI is deployable on any OS that has Node.js and mongoDB binaries available. This includes the cited 
			architectures and OS, as well as the ARMv7 architecture of the Raspberry Pi 2. HRUI has been tested on Linux 
			Mint 17-17.2 64-bit, Windows 7/8.1/10 and Arch Linux ARMv7. OS X was not tested for lack of a machine that uses 
			the operating system, but should work the same as the Linux distribution. The only caveat to this point is that 
			Media streaming is not available on Windows, and hasn't been tested on OS X (presumably will not work without 
			significant changes).

		\end{itemize}
	\item \textbf{Open Source}: \textit{All of the software APIs, Libraries and Frameworks that are used in the application 
	must be open source. This generally entails that the software is free to use, study and modify, for any purpose. See 
	section \ref{opensourcemovement} of this report for more on open source software. The tools used for development (IDEs, 
	OS, Hardware etc.) need not be open source, but will be preferable if available}.\\

	All the software APIs, libraries and frameworks used in HRUI are completely open source. See section 
	\ref{opensourcemovement} for a brief discussion on the open source movement.
\end{itemize}

Therefore, the project developer personally believes that the project has been a success. The concepts and technologies 
learned throughout development have been a source of academic growth and satisfaction, as well as a primer on the tools 
used in the modern tech industry, and the challenges of software engineering.\\
\newpage
\section{Open Source Movement} \label{opensourcemovement}
Open Source is a production and development concept that arose in the software development community circa 1980, that has 
been extended to all production and development industries, that states that the design, blueprint or recipe for a product, 
called \textbf{source}, should be \textbf{freely available to use, modify, study and integrate in other products by anyone, 
for any purpose.}\\

The Open Source movement is a social movement of individuals that are committed to the proliferation of open source 
products for as many applications as possible. Since this is a software project, the following discusses that particular 
aspect of open source, but it has extended to hardware design, manufacturing, industrial specifications etc. In fact, the 
Crazyflie 2 used in this project as a proof-of-concept (see section \ref{crazyflie2}) is an open source drone: the design 
documents and manufacturing processes are all freely available. The history of the movement is too long and off-topic to 
recount, but suffice it to say, since the beginning of the expansion of information technologies, the open-source movement 
has existed in some way or another. Some figures of the movement are well-known and respected in the software development 
community: Richard Stallman (1953), founder of the GNU Project in 1983, one of the most important open-source projects that 
still exist to this day, or Linus Torvalds (1969), responsible for the development of the Linux kernel, and developer of 
the git revision control system. But the true power of the movement is the willingness of thousands of individuals to 
dedicate their free time, without any expected return on investment, to create software that is free to use, modify and 
study by anyone. The only thing sustaining the foundations that are dedicated to free software are donations and the will 
of software enthusiasts. Some of these foundations are:
\begin{itemize}
	\item Apache Software Foundation: Responsible for dozens of projects, most notably the most used web server in the 
	world, the Apache HTTP server.
	\item Linux Foundation: Manages the community of developers that make the kernel evolve day by day. Linux powers most 
	special purpose computers in the world today: super computers, web servers, embedded computers etc. As well as being a 
	perfectly suitable general purpose OS, and the basis for millions of Android phones worldwide.
	\item GNU Project: Maintains the GNU/Linux project, an OS with open-source only software.
	\item Creative Commons: Holders of the Creative Commons copyright license.
	\item Wikimedia Foundation: parent foundation of the free on-line encyclopedia, Wikipedia. 
\end{itemize}
Out of the movement, different open-source licenses have been put forward. An open source license is a piece of text 
attached to the software that states the conditions by which the user has the right to modify, use and study the source 
freely, while also enforcing that these rights are upheld. These conditions normally entail attribution of credit to the 
original author, and the redistribution of the license, attached to the source. Some popular open-source licenses:
\begin{itemize}
	\item Apache 2.0 License: Only requires preservation of the copyright notice and disclaimer.
	\item GNU General Public License (GPL) 3.0: This license obligates derivative works to also be open-source. This is 
	called ``copyleft''.
	\item MIT License: Originated at the Massachusetts Institute of Technology, allows propietary software to use the work 
	without having to relinquish it's proprietary status. This license is very popular for it's simplicity and is used for 
	almost all of the external software in this project. The license is included in the source code of HRUI (at the end of 
	index.html), as per the license conditions. The following is the MIT License.\\
\end{itemize}
Copyright (c) <year> <copyright holders>\\

Permission is hereby granted, free of charge, to any person obtaining a copy of this software and associated documentation 
files (the "Software"), to deal in the Software without restriction, including without limitation the rights to use, copy, 
modify, merge, publish, distribute, sublicense, and/or sell copies of the Software, and to permit persons to whom the 
Software is furnished to do so, subject to the following conditions:\\

The above copyright notice and this permission notice shall be included in all copies or substantial portions of the 
Software.\\

THE SOFTWARE IS PROVIDED "AS IS", WITHOUT WARRANTY OF ANY KIND, EXPRESS OR IMPLIED, INCLUDING BUT NOT LIMITED TO THE 
WARRANTIES OF MERCHANTABILITY, FITNESS FOR A PARTICULAR PURPOSE AND NONINFRINGEMENT. IN NO EVENT SHALL THE AUTHORS OR 
COPYRIGHT HOLDERS BE LIABLE FOR ANY CLAIM, DAMAGES OR OTHER LIABILITY, WHETHER IN AN ACTION OF CONTRACT, TORT OR OTHERWISE, 
ARISING FROM, OUT OF OR IN CONNECTION WITH THE SOFTWARE OR THE USE OR OTHER DEALINGS IN THE SOFTWARE.\\

The open source movement is essential for software to remain in the hands of individuals. With no open source, software 
development would be exclusively in the hands of companies that are inherently designed to maximize their own profit, even 
at the expense of the user. Without open source there's no alternative, free, way to access software that is required to 
access the wealth of information provided by the Internet, that since it's inception is guaranteeing that anybody has the 
chance to access any knowledge, without any gate keepers.\\

At a much more menial level, this project wouldn't be possible without open source. But imagine how many other projects 
like this one would be doomed before even being thought of, for lack of financing required to develop or license all the 
proprietary software required for it's development. The thought becomes much less trivial.\\
\begin{center}
\vfill
\line(1,0){250}\\
\textit{\textbf{Thank you for taking the time to read this report, and thank you Ricardo, for your time and patience.}}
\end{center}
