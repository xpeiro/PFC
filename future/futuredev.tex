% 	HTML5 Robot User Interface Project Report: Future Development
% 	An ASLab Project,
% 	Developed by Daniel Peiró
% 	ETSII, UPM 2014-2015
\chapter{Future Development} \label{futuredevelopment}
As with all software engineering projects, and more generally any engineering project, work is not finished once the final 
product is handed over to the client. Specifically for software development, and software projects that follow the waterfall 
design process (briefly explained in chapter \ref{planning}) like this project, after the system is designed and implemented, 
the project enters it's last phase, called maintenance.\\
\section{Maintenance and Support}
This phase consists of four types of maintenance:
\begin{itemize}
	\item \textbf{Corrective Maintenance}: Modifications to the system to correct errors or issues that arise with the use of 
	the system. This is prevented in the previous step in the design process, verification, but inevitably errors or software 
	bugs will slip through, given that it's virtually impossible to go through all the possible variations in the usage of the 
	system.
	\item \textbf{Adaptive Maintenance}: Upgrades to the software due to changes in the technology it uses as well as 
	significant changes to the industry standards. This type of maintenance is briefly discussed in further sections, and for 
	HRUI basically entails: scalability (section \ref{scalability}) and new additions to the standard (section 
	\ref{newadditionstothestandard}).
	\item \textbf{Perfective Maintenance}: Adapts the system even closer to the client requirements if possible, adding new 
	features or options that weren't taken into account in the design process. This type of maintenance indicates that the 
	requirements were not accurately captured from the client. There'll always be a certain level of uncertainty in this regard 
	and some requirements may evolve over time and develop under use of the system. However, if this maintenance is extensive, 
	it could indicate issues in the design process, since it was unable to provide the solution to the client's requirements out 
	of the gate.
	\item \textbf{Preventive Maintenance}: Predicts and tries to solve future problems that may arise. This is similar to 
	adaptive maintenance, except that in this case, the upgrades are done preemptively, and in the other they're made once the 
	change is made evident. 
\end{itemize}
The plan for HRUI's maintenance and support is divided into three phases:
\begin{enumerate}
	\item \textbf{Developer Usage and feedback}: The application is released to robot programmers and researchers, with the 
	objective of analyzing the strengths and weaknesses of the application from the back-end, controller programming perspective 
	of the application. Any use-cases that weren't taken into account in initial development are then evaluated taking into 
	account two variables:
		\begin{itemize}
			\item Frequency: If the use-case is present in a wide array of applications for the software or not.
			\item Cost: The amount of time, effort, and if applicable, monetary cost of implementing the use-case.
		\end{itemize}
	These variables are pondered, in such a way that a use-case that is very costly and is not used frequently will probably be 
	discarded, while a use-case that is used very frequently and requires little effort will most likely be implemented. Cases 
	in between the extremes will be analyzed further. This phase will likely entail perfective and corrective maintenance.
	\item \textbf{User Experience and Feedback}: The application, once all of the developer feedback has been dealt with, is 
	rolled out to users, that will experience the application from the front-end. In reality, these users will most likely be 
	the robot programmers and researchers themselves at this stage, but the use of the application is decidedly divided due to 
	it's double decoupling, explained in section \ref{mvcpattern}. This phase will likely entail a large amount of corrective 
	maintenance, as bugs will be discovered when using the application in all the different permutations possible.
	\item \textbf{Public Roll-out and Application Maturity}: The application is then rolled out to the public, generally bundled 
	with the robot controller and the actual robot it was designed for. This will not be the main use of the application as it 
	is inherently designed to be used as a tool to make robot research easier, not as a commercial product. Nevertheless, the 
	application is a perfectly valid interface that could be bundled with a robot (some robots come with much less sophisticated 
	controllers, and charge a lot of money for them). This would likely entail extensive security maintenance that falls into 
	preventive and corrective maintenance categories alike, as well as any marginal bug that slipped through the previous phases.
\end{enumerate}
It should be noted that this phase of the project, particularly perfective maintenance tasks are made decidedly simpler thanks 
to the implementation of the overall MVC architecture, and the modular front-end architecture. These architectures are both 
discussed in chapter \ref{systemarchitecture}. By using these, as explained in the cited section, adding features to the 
application is reduced to three confined steps, that do not require the modification of previously existing modules and 
controllers. These steps are:
\begin{itemize}
	\item Adding a controller: a piece of code that contains the business logic that the feature requires.
	\item Allocating resources in the model: an ``item'' in the model that contains the state of the feature's variables.
	\item Designing a view: a visual representation and input method for the user to request actions from the controller, and 
	retrieve data from the model.
\end{itemize}
This of course is a broad stroke, a generalization, and some modifications might entail restructuring the application in some shape 
or form, but in general adding or subtracting features will remain a relatively isolated affair.\\

Of course, this maintenance planning is theoretical, given that the application is part of an academic project.
\section{Scalability} \label{scalability}
One of the more interesting ways that the application could be modified, as part of adaptive maintenance explained in the previous 
section, is by adding the possibility of having an N amount of users on the front-end at any given time using the application. This 
of course is possible as is, but all users are considered ``the same user'', as all the inputs are handled as if there were only 
one user, i.e. there is only one joystick item in the application model, no matter how many users are manipulating their own 
instance of the client, they will only modify the one set of coordinates. Imagine then that for every user, a new joystick item was 
created holding the state of every joystick individually, and that a robot programmer could get hundreds if not thousands of unique 
inputs. The possibilities with such amount of inputs is endless.\\

This adaptation would be very interesting, but would obviously negate the portability and lightweight nature of the application as 
is. It would require the server to \textit{scale horizontally}, meaning that the model, the mongoDB database, would have to be run 
on multiple machines, and to be able to handle a large amount of clients, the resources of multiple machines would have to be 
pooled (this is somewhat alleviated by the asynchronous and non blocking I/O of the node.js framework, see section \ref{nodejs} for 
more). This is not an issue, given the distributed nature of the mongoDB DBMS, and the fact that scalability would be progressive: 
the application could still work on a Raspberry Pi (as it does as is) with a low number of concurrent users, given that the 
workload would scale with the amount of users.\\

Once the infrastructure was ready, the actual changes to the code would not be too dramatic. The system changes would probably 
entail something in this order:
\begin{enumerate}
	\item On client connection, a unique, random, id number is generated identifying the client unequivocally.
	\item This id is sent to the back-end, where a new collection is spawned in the model, the mongoDB database, with that id 
	attached to it. Currently there is only one collection in the database, which holds all the items that hold the application 
	state. This collection would be cloned with each client connection, identified by the id given in the previous step. Each 
	client would have it's own application state, stored independently.
	\item All data packages that are delivered to and from the server, would be attached with the id of the client that generated 
	the event or the id of the collection from where it was extracted from the model. This would be trivial, since all the 
	communications have been isolated through a single controller, all the controller would have to do is ``label'' the packet 
	before sending it, and reading its label on reception.
	\item On client disconnection, either the collection in the model is deleted to free resources, or is labeled as inactive, so 
	the programmer can decide what to do with the outdated data.
\end{enumerate}
The best part of this is that robot controllers would be as easy to design with one client as with N clients. The same method is 
used to get data from the database, regardless of how many collections and items are present in it. A controller could do a 
collection-wide poll on the state of N joysticks, and use that to, for example, control a robot swarm, or to control the different 
parts of a very complex robot. With the custom input module (see section \ref{custominput}), this could be extended to any input, 
period. The possibilities are enormous.\\

This scalability is already available on the back-end of the application. Since the mongoDB DBMS is inherently distributed, there's 
no limit to the amount of client robot controllers that could access the inputs on different machines, that could have any hardware 
capabilities necessary, without loading the server, that could remain a low end computer or a micro computer like the Raspberry Pi, 
given that the only processing it requires would be networking. Live streaming media could be an issue, given that it relies on the 
servers resources, and accesses the servers device files, but a relay stream could be devised to feed into a ``mock'' device file 
on the server, or more directly, change the media acquisition method to a distributed one (Avconv has an integrated networking 
server, so this would probably be the simplest method).
\section{New Additions to the Standard} \label{newadditionstothestandard}
Another interesting aspect of adaptive maintenance (as explained at the beginning of this chapter) for HRUI is the fact that the 
HTML5 Standard is still evolving, and in the future HTML.next (the name the W3C is giving the next HTML specification after HTML5) 
will add new features that might be useful to make the interface more usable, more attractive and more efficient so that it can be 
available on an ever expanding array of devices.\\

There are two ways that this can affect HRUI:
\begin{itemize}
	\item \textbf{Change in experimental implementations}: This applies to those APIs that HRUI uses that are still non-standard or 
	have been implemented in different ways by different browser developers. This is the case for the Web Speech API and the 
	Gamepad API, used by the voice commands (section \ref{voicecommands}) and gamepad (section \ref{gamepad}) modules respectively. 
	In the case of the Speech API, it's only available on webkit based browsers, e.g. Google Chrome, and while it probably will be 
	implemented eventually by all browsers, the implementations may differ, and consequently a new standard will be developed in 
	concert between all vendors. Or not. The fact is it's subject to change, and it will be necessary to adapt the module when 
	changes arise. In the case of the Gamepad API, it has been implemented by the three major browser developers, but in different 
	ways. Particularly, the mapping of buttons and if it's required for the user to press a button on the controller for initial 
	detection. The specification recommends a layout for the buttons, so in the future the developers should converge on the 
	recommendation. Again, or not. Once the API is mature enough, HRUI will have to be adapted to it. To a lesser extent, it's 
	possible that other, more stable specifications change in the future, but the solution would be the same as with experimental 
	APIs, and it's not likely in any case.
	\item \textbf{New additions in future standard}: It's possible and very likely that in the future, new additions to the 
	standard will provide functionality that may be useful for HRUI. Some of the proposals made can be found in the 
	\href{http://www.w3.org/wiki/HTML/next}{HTML.next WIKI} on the W3C website. One of the more interesting proposals is adaptive 
	streaming or as the W3C now calls it, \href{https://w3c.github.io/media-source/\#mediasource}{``Media Source Extensions''}, 
	that could greatly enhance the Live Video module. Or the \href{http://www.w3.org/wiki/HTML/next#.22behaviors.22}{behaviors} 
	object for JavaScript that could simplify event capturing and binding. The developer for one, would like a definitive standard 
	for media formats and client decoders. All of this would be progressively added, if the use-cases justify it.
\end{itemize}
It should be noted again that these modifications are made decidedly simpler thanks to the implementation of the overall MVC 
architecture, and the modular front-end architecture. These architectures are both discussed in chapter 
\ref{systemarchitecture}. By using these, as explained in previous sections and the cited chapter, adding features to the 
application is reduced to designing very simple, isolated components, that do not require the modification of previously 
existing modules and controllers.\\

HTML has a history of being very backwards compatible, and efforts to break with this trend have been generally unsuccessful. 
See sections \ref{HTML} and \ref{HTML5} for more. Because of this, it's very unlikely that this application will become obsolete 
by becoming incompatible with devices or browsers. Obsolescence will like come from better and newer solutions becoming 
available as technology progresses naturally.